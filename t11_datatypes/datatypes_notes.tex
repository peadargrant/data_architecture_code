\chapter{Datatypes}
\label{ch:datatypes}

\section*{Recommended reading}

By far the best resource is the PostgreSQL manual \href{https://www.postgresql.org/docs/current/datatype.html}{Chapter 8}.
You should be using it regularly!

\section{Numeric data}

Numeric data largely falls into the categories of: integer, fixed precision and arbitary precision data.

\subsection{Integers}

Integers have an allowable range according to the datatype.
\begin{itemize}
\item Generally choose from \texttt{smallint, int, bigint} depending on the expected value range. 
\item \texttt{bigint} may suit incrementing ids in long-running systems.
\end{itemize}

\subsection{Floating point}

\subsection{Fixed precision}

\section{Textual data}

Recommend to use \texttt{text} only.
Use contraints to control length if required. 

\section{Good practices}

\begin{enumerate}
  
\item Do not store numbers as text.

\item Do not store boolean true / false ( or any synonyms ) as text.

\item Do not sure enumerated types as text. Either use an ENUM and/or a foreign-keyed table.

\item Anything where you need absolute precision after the decimal point must be NUMERIC, not FLOAT. 
  
\item Do not be tempted to store a single logical date/time as separate date and time columns.
  Always use a single timestamp for this. 
  
\end{enumerate}

