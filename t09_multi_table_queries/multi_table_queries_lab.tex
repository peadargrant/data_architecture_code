\chapter{Multi-table Queries lab}

\begin{enumerate}

\item Use \texttt{psql} to connect to the \texttt{france} database.

\item Use the client command \texttt{\textbackslash d} to describe all the database objects.

\item Use the \texttt{\textbackslash d towns} command to describe the \texttt{towns} table.
  Do the same for the other two tables.
  From the foreign key / referencing information, draw the relationship between the tables on paper. 

\item Write a query to count the number of towns.
  Similarly the number of regions and departments. 
  
\item Construct a query to summarise the number of towns, regions and departments in a single result set.
  Use the \texttt{UNION} clause to combine similar result sets.
  Use the \texttt{AS} clause to name columns as needed.
  The output should be:
\begin{verbatim}
    entity    | count
 -------------+-------
  towns       | 36684
  regions     |    26
  departments |   100
  (3 rows)
\end{verbatim}

\item Construct a query to list the regions by id ascending showing only the \texttt{id}, \texttt{code} and \texttt{name} for regions with \texttt{id} less than 7, ordered by \texttt{region\_name}.
  You will need to use an \texttt{ORDER BY} clause.
  Output must be exactly:
\begin{verbatim}
 region_id | region_code |    region_name
-----------+-------------+-------------------
         6 | 21          | Champagne-Ardenne
         1 | 01          | Guadeloupe
         3 | 03          | Guyane
         5 | 11          | Île-de-France
         4 | 04          | La Réunion
         2 | 02          | Martinique
(6 rows)
\end{verbatim}
  \label{step:region-list}

\item Write down the column names that link the \texttt{departments} and \texttt{regions} tables.

\item Modify the query from Step~\ref{step:region-list} to show the number of departments in each region.
  You will need to use a \texttt{JOIN} clause.
\begin{verbatim}
 region_id | region_code |    region_name    |  department_name
-----------+-------------+-------------------+-------------------
         6 | 21          | Champagne-Ardenne | Ardennes
         6 | 21          | Champagne-Ardenne | Aube
         6 | 21          | Champagne-Ardenne | Haute-Marne
         6 | 21          | Champagne-Ardenne | Marne
         1 | 01          | Guadeloupe        | Guadeloupe
         3 | 03          | Guyane            | Guyane
         5 | 11          | Île-de-France     | Essonne
         5 | 11          | Île-de-France     | Hauts-de-Seine
         5 | 11          | Île-de-France     | Paris
         5 | 11          | Île-de-France     | Seine-et-Marne
         5 | 11          | Île-de-France     | Seine-Saint-Denis
         5 | 11          | Île-de-France     | Val-de-Marne
         5 | 11          | Île-de-France     | Val-d'Oise
         5 | 11          | Île-de-France     | Yvelines
         4 | 04          | La Réunion        | La Réunion
         2 | 02          | Martinique        | Martinique
(16 rows)
\end{verbatim}
\label{step:regions-and-departments}
  
\item
  Starting with your query from Step~\ref{step:regions-and-departments}, modify so that it counts the number of departments per region.
  You will need use the \texttt{GROUP BY} clause.
  Output should be exactly:
\begin{verbatim}
 region_id | region_code |    region_name    | number_of_departments
-----------+-------------+-------------------+-----------------------
         6 | 21          | Champagne-Ardenne |                     4
         1 | 01          | Guadeloupe        |                     1
         3 | 03          | Guyane            |                     1
         5 | 11          | Île-de-France     |                     8
         4 | 04          | La Réunion        |                     1
         2 | 02          | Martinique        |                     1
(6 rows)
\end{verbatim}
\label{step:number-of-departments-per-region}

\item
  Modify the query from Step~\ref{step:number-of-departments-per-region} to exclude regions with less than 2 departments.
  You will need to use a \texttt{HAVING} clause.
  Output should be exactly:
\begin{verbatim}
 region_id | region_code |    region_name    | number_of_departments
-----------+-------------+-------------------+-----------------------
         6 | 21          | Champagne-Ardenne |                     4
         5 | 11          | Île-de-France     |                     8
(2 rows)
\end{verbatim}
\label{step:number-of-departments-per-region-not-1}

\item
  Modify the query from Step~\ref{step:number-of-departments-per-region-not-1} so that it shows instead the number of towns in each region from the entire set of regions, ordered to show regions with the highest number of towns first.
  You will need an additional \texttt{JOIN} clause.
  Output should be exactly: 
\begin{verbatim}
 region_id | region_code |     region_name      | number_of_towns
-----------+-------------+----------------------+-----------------
        20 | 73          | Midi-Pyrénées        |            3020
        22 | 82          | Rhône-Alpes          |            2879
        13 | 41          | Lorraine             |            2339
        19 | 72          | Aquitaine            |            2296
         7 | 22          | Picardie             |            2292
        11 | 26          | Bourgogne            |            2046
         6 | 21          | Champagne-Ardenne    |            1948
         9 | 24          | Centre               |            1842
        10 | 25          | Basse-Normandie      |            1812
        15 | 43          | Franche-Comté        |            1786
        12 | 31          | Nord-Pas-de-Calais   |            1546
        24 | 91          | Languedoc-Roussillon |            1545
        16 | 52          | Pays de la Loire     |            1502
        18 | 54          | Poitou-Charentes     |            1464
         8 | 23          | Haute-Normandie      |            1420
(15 rows)
\end{verbatim}
  
\end{enumerate}



