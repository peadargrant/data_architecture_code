\chapter{Single Table}

\section{Introduction}
\label{sec:introduction}

This week's class covers the setup and usage of a basic database.
As a data analyst you will often encounter situations where you will run queries on databases that already exist.
Other times you will be tasked with setting up a database to conduct one-off or ongoing analyses.

We will therefore look at how data needs to be modelled for a single table, focusing on data types and constraints to improve data integrity.

You must closely refer to the
\href{https://www.postgresql.org/docs/15/ddl.html}{Data Definition} and
\href{https://www.postgresql.org/docs/15/dml.html}{Data Manipulation}
chapters of the
\href{https://www.postgresql.org/docs/15/index.html}{PostgreSQL manual} through this week's exercises.

As this is a MSc-level course it is expected that you will be able to construct SQL statements by researching documentation for yourself!
The best way to learn is by trying!

\section{Prerequisities}
\label{sec:prerequisities}

Before starting make sure that you:

\begin{itemize}
\item
  created your SSH public/private key-pair.
\item
  can login over SSH to the shared server.
\item
  reviewed the simple \texttt{SELECT} queries from last week.
\end{itemize}

For today's lab login to the shared server and run \texttt{psql}. You
should be connected to a database with the same name as your username.
See
\href{https://www.postgresql.org/docs/14/tutorial-accessdb.html}{Accessing
a Database} in the PostgreSQL manual.

\section{Key demonstration concepts}\label{key-demonstration-concepts}

Key concepts will be easily learned through repeated practice and
reference to manual.

\subsection{Key DDL concepts}

\begin{description}
\item[CREATE table]
  to define table name, columns (type, constraints)
\item[ALTER table]
  for incremental changes.
\item[DROP table]
  without warning!
\end{description}


\subsection{Column names}

Following a consistent and workable pattern with column names will help a lot:
\begin{enumerate}
\item names should be descriptive
\item try to keep names reasonablly short
\item do not use spaces, use the underscore instead (\_)
\end{enumerate}
Column names can be changed afterwards.

\subsection{Datatypes} 

Choose datatypes correctly for data of interest\\
\url{https://www.postgresql.org/docs/13/datatype.html}

Generally changing a datatype is incredibly difficult after the fact!


\section{Constraints}
\label{sec:constraints}

Constraints allow us to define acceptable values for columns to aid data
integrity.

\href{https://www.postgresql.org/docs/14/ddl-constraints.html}{Refer to
constraints section in the PostgreSQL manual} for: check, not null,
unique, primary key. \emph{We will meet foreign keys and exclusion
constraints later on}

\subsection{NOT NULL}\label{not-null}

The NOT NULL constraint prevents a particular column value being null.
Writes will fail if set to NULL or if unspecified unless default column value specified.

\begin{verbatim}
-- During table creation:
CREATE TABLE tasks (
   description text not null,
   /* other columns */
);
\end{verbatim}


Whether NULL should be allowed depends on problem of interest.
By default NULL is allowed.
Must specify NOT NULL if not.

Can modify the column's NULLable status afterwards:
\begin{verbatim}
-- make a column NOT NULL
ALTER TABLE tasks ALTER COLUMN description SET NOT NULL

-- remove the NOT NULL constraint from a column
ALTER TABLE tasks ALTER COLUMN description DROP NOT NULL
\end{verbatim}


\subsection{UNIQUE}\label{unique}

The UNIQUE constraint prohibits two rows from having an equal set of
attribute values for the columns specified:

\begin{itemize}
\item Use UNIQUE after column definition if column should be unique
\item Use UNIQUE(col1, col2) if two or more columns should be unique after the column definitions
\end{itemize}


\begin{verbatim}
-- During table creation:
CREATE TABLE tasks (
   description text unique,
   /* other columns as necessary */
   project_code text,
   subproject_code text,
   unique(project,subproject)
); 

-- Afterwards
ALTER TABLE tasks ADD UNIQUE (project, subproject);
\end{verbatim}

\section{Default values}
\label{sec:default-values}

Default values are automatically inserted when unspecified in INSERT
statement.
Often used in conjunction with NOT NULL.
Can be static value or based on functions (see later).

\begin{verbatim}
-- when creating a table
CREATE TABLE accounts (
    /* other columns */
    balance bigint not null default 0;
    created timestamp not null default now(); 
);

-- afterwards
ALTER TABLE tasks ALTER COLUMN description SET default 'xyz'
ALTER TABLE tasks ALTER COLUMN description DROP default
\end{verbatim}


\section{Primary key}
\label{sec:primary-key}

Primary key is a unique key that is not null.
Unambiguously identifies each row.

Every table should have 1 primary key.
Cannot have more than 1. 
Most DBMS do not enforce this, but you should consider it mandatory.
Use PRIMARY KEY if column is the primary key. 

\begin{verbatim}
-- specifying primary key
CREATE TABLE accounts ( 
    account_number bigint primary key
    /* other columns as necessary */
);

-- primary key covering two columns
CREATE TABLE accounts (
    branch bigint, -- not null automatic
    account_number bigint,  -- not null automatic
    /* other columns */
    primary key (branch, account_number)
    /* other constraints etc. */
)
\end{verbatim}



\subsection{Auto-increment column}
\label{sec:auto-increment-column}

We can use the column types SERIAL or BIGSERIAL for auto-incrementing columns.
Generally prefer BIGSERIAL.
Shorthand for default value based on sequence generator (later on).

\begin{verbatim}
CREATE TABLE customers ( 
    id BIGSERIAL PRIMARY KEY,
    /* other columns & constraints */
);
\end{verbatim}



\section{Transaction control}
\label{sec:transaction-control}

\href{https://www.postgresql.org/docs/15/tutorial-transactions.html}{PostgreSQL
transaction control} can be used in simplest form to give basic ``undo''
capability:

\begin{verbatim}
-- start a new transaction
BEGIN;

/* execute SQL statements then either: */

COMMIT; /* save the changes */
-- or
ROLLBACK; /* undo the changes */
\end{verbatim}

\subsection{Error handling}
\label{sec:error-handling}

If we don't BEGIN a transaction we implicitly BEGIN before the statement
and COMMIT after it. Errors (e.g.~syntax) will abort the transaction,
requiring a ROLLBACK before any more statements will be accepted.

\subsection{Savepoints}\label{savepoints}

\begin{verbatim}
BEGIN;

/* sql statement block 1 */

-- define a savepoint
SAVEPOINT sp1;

/* sql statement block 2 */

-- rollback to the savepoint undoes block 2; 
ROLLBACK TO sp1;

-- then either:
COMMIT; /* or ROLLBACK */
\end{verbatim}

\section{DML operations}

\subsection{DML operations}

\begin{description}
\item[INSERT]
  rows populating specified columns with data given.
\item[UPDATE]
  table, optionally select rows with \texttt{WHERE}.
\item[DELETE]
  rows from a table, optionally select rows with \texttt{WHERE}.
\end{description}



\section{Lab exercise}

\begin{enumerate}
\def\labelenumi{\arabic{enumi}.}
\item
  Setup a simple table to keep track of your tasks / expenses / other
  data. Your table should at least have a description and due date. Add
  in other columns (such as priority, e-mail to notify when done) as
  necessary.

  \begin{itemize}
  \item
    Keep a track of which columns need to be there.
  \item
    Make clear decisions around NULL / NOT NULL.
  \item
    Decide what column(s) is/are to make up the primary keys.
  \end{itemize}
\end{enumerate}




