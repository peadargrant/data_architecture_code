\chapter{Multi-table Databases lab}
\label{ch:multi-table-databases-lab}

Consider a simple task manager application using two tables for projects, tasks (linked to projects) and notes (linked to tasks)
\begin{description}
\item[Projects] have auto-incrementing ID, name, creation timetstamp.
\item[Tasks] have auto-incrementing ID, description, project, due date, creation timestamp, status (pending, in-progress, complete).
\item[Notes] have auto-incrementing ID, task, note, creation timestamp
\end{description}

\begin{enumerate}
\item On paper, plan out the table definitions.
  \begin{enumerate}
  \item Every table should have appropriate data types, NULL/NOT NULL, PRIMARY KEY, UNIQUE KEYs.
  \item You should try to make use of enumerated types and domains.
  \item Tables should be linked using foreign keys with correct ON DELETE / ON UPDATE behaviours chosen.
  \end{enumerate}
\item Implement your design as tables.
\item Insert test data for at least 3 projects with 3 tasks each. 
\item Create a view that encapsulates a query involving at least two tables and provides at least one computed column using the CASE clause. 
\end{enumerate}



