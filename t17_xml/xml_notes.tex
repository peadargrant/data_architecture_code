\documentclass[slides]{pgnotes}

\title{XML}
\newcommand{\entityref}[1]{\&#1;}

\tikzstyle{cpt} = [fill=black, text=white, draw=none, rounded corners, rectangle, minimum width=2.5cm, minimum height=1.5cm]
\tikzstyle{ml} = [draw=black, ultra thick, rounded corners, rectangle, minimum width=2.75cm, minimum height=1.5cm] 
\tikzstyle{subset} = [draw=black, ultra thick, ->, dashed] 
\tikzstyle{application} = [draw=black, ultra thick, ->] 
\tikzstyle{bizr} = [anchor=south, text=white] 


\begin{document}

\maketitle

% \section*{Aims}

% \begin{enumerate}
%   \item Explain the essential structures and rules of an XML document.
%   \item Distinguish well-formed XML from non-well-formed XML. Use tools to diagnose non-well-formed XML.
%   \item Construct vocabularies to capture structured information using XML.
% \end{enumerate}


\section{eXtensible Markup Language}

Extensible Markup Language (XML) is a \textbf{markup} language that defines a set of rules for encoding documents in a format that is both human-readable and machine-readable.
It's a subset of SGML, of which HTML is also an \textit{application}. 
The tag names are up to you. 

\subsection{SGML}

Standard Generalised Markup Language (SGML) originated as a means to store and transfer US government and military documents in machine-readable form, over a timespan of several decades.
SGML uses \textbf{tags} to mark up a document.
One of the best known examples of SGML is HTML.

\inputminted{html}{htmlpage.html}

For many applications, SGML was too complicated, and so XML was designed.
Apart from HTML and Document Type Declarations, you are unlikely to encounter SGML in its non-XML form very frequently.
XML documents themselves are valid SGML, but have additional restrictions on their layout.
(SGML documents are not valid XML.)
 

\subsection{Sample XML document}

\inputminted{xml}{intro.xml}

\subsection{Design Goals}

\begin{enumerate}
\item XML shall be straightforwardly usable over the Internet.
\item XML shall support a wide variety of applications.
\item XML shall be compatible with SGML.
\item It shall be easy to write programs which process XML documents.
\item The number of optional features in XML is to be kept to the absolute minimum, ideally zero.
\item XML documents should be human-legible and reasonably clear.
\item The XML design should be prepared quickly.
\item The design of XML shall be formal and concise.
\item XML documents shall be easy to create.
\item Terseness in XML markup is of minimal importance.
\end{enumerate}

 

\subsection{Benefits of XML}

\begin{enumerate}
\item You define the \textbf{vocabulary}, so it provides flexibility.
\item XML is \textbf{simple} and used widely.
\item Machine and \textbf{human readable}.
\item \textbf{Lots of APIs} and parsers to handle XML. 
\item XML Data \textbf{can be checked} for correctness.
\end{enumerate}

\subsection{Issues with XML}

\begin{enumerate}
\item XML documents require more storage than fixed length or comma delimited files.
  \begin{itemize}
  \item But, they compress well!
  \end{itemize}
\item Can help but not does not solve the politics of competing formats (multiple vocabs).
\item Definitions can be overly nested and made too complex.
\end{enumerate}
 
\section{Structure of an XML doc}

XML has a Directed Acyclic Graph tree structure, built from \textbf{nested} tree of \textbf{Elements} delimited by \textbf{tags}.
Elements are \textbf{case-sensitive}.

\inputminted{xml}{closed.xml}

\subsection{Nesting rules} 

\begin{itemize}
\item Elements nested inside a \textit{parent} element are called \textit{child} elements. 
\item Elements must be nested correctly. Child elements must be enclosed within their parent elements.
\item All elements must be nested within a single document or root element.
\item There can be only one \textbf{root} element in a document.
\item Indentation, while helpful, is just visual.  
\end{itemize}


\subsection{XML declaration}

The XML declaration is always the first line of code in an XML document, and tells the \textbf{processor} that what follows is XML.
It can also provide information about how the parser should interpret the document.

For now, the version is always 1.0.
This attribute allows the program (or person) reading the document to read older versions if specifications change. 


\subsection{XML comments}

Comments may appear anywhere after the declaration.
The syntax for comments is:

\inputminted{xml}{comment.xml}

Two or more dashes should not appear one after the other in a comment.


\subsection{Free-format}

XML documents are free-format.
For example, the document:

\inputminted{xml}{free-format-condensed.xml}
  
is the exact same as:
  
\inputminted{xml}{free-format-expanded.xml}


\section{Attributes}

An attribute is a feature or characteristic of an element. Attributes are text strings and must be placed in single or double quotes. 
Use that which makes most sense in deciding whether to store the main data in an element or attribute.
Often an element is a better choice.

\inputminted{xml}{attribute.xml}


\subsection{Mixing elements and attributes}

\inputminted{xml}{mixing.xml}



\section{Well-formed XML}

When XML text satisfies the syntactic rules as laid out in the XML specification, we say that it is well-formed \citep{goldfarb:2003:the-xml-handbook}.

\subsection{Criteria}

\begin{enumerate}
\item All XML elements must have a closing tag, except empty elements. 
\item Tags are case sensitive.
\item All elements must be properly nested.
\item Documents must have one single root element.
\item Attribute values must be quoted.
\item Certain characters must be replaced with \textit{Entity References}. 
\end{enumerate}


\subsection{Entity references}

Entity references replace certain characters as detailed in \autoref{tab:entity-reference-xml}.

\begin{table}[htbp]
  \centering
  \begin{tabular}{l l l}
    \toprule
    \textbf{Normal symbol} & \textbf{Meaning} & \textbf{Replace with} \\
    \midrule
    $<$ & less than & \entityref{lt} \\
    $>$ & greater than & \entityref{gt} \\
    \& & ampersand & \entityref{amp} \\
    ' & apostrophe & \entityref{apos} \\
    '' & quotation mark & \entityref{quot} \\
    \bottomrule
  \end{tabular}
  \caption{Entity References}
  \label{tab:entity-reference-xml}
\end{table}


\subsection{Non well-formed XML}

Non well-formed XML will cause errors in applications and parsers that expect strict XML syntax.

\subsection{Line termination}

By convention, different OSes use different delimiters to signal the end of a line in a text file, \autoref{tab:newline}.
XML specifies that new lines are \textbf{always} stored as LF, regardless of the operating system. 

\begin{table}[htbp]
  \centering
  \begin{tabular}{l l}
    \toprule
    \textbf{Operating system} & \textbf{Delimiters} \\
    \midrule
    Windows & CR LF \\
    Unix, Mac OS X & LF \\
    Mac (System 9 and below) & CR \\ 
    \bottomrule
  \end{tabular}
  \caption{Newline characters}
  \label{tab:newline}
\end{table}

\subsection{Checking for well-formed XML}

The \texttt{xmllint} utility can check XML well-formedness.

\begin{minted}{bash}
# consult manpage
man xmllint

# basic usage
xmllint filename.xml 
\end{minted}
 
  








\section{Vocabulary}

Although the structure and syntax of XML documents is rigorously specified, you are free to choose a suitable vocabulary of element and attribute names for the application of interest.
Some common vocabularies are listed in \autoref{tab:common-vocabularies}. 

\begin{table}[htbp]
  \centering
  \begin{tabularx}{1.0\linewidth}{l X}
    \toprule
    \textbf{Vocabulary} & \textbf{Description} \\
    \midrule
    XHTML & a variant of HTML that complies with XML restrictions \\
    SOAP & message format for web services \\
    OpenOffice XML & Microsoft's current office word-processing format \\
    Opendoc & standard formats for (non-Microsoft Office) office documents \\
    Docbook XML & markup for writing (primarily) technical documentation  \\
    \bottomrule
  \end{tabularx}
  \caption{Common standardised XML vocabularies}
  \label{tab:common-vocabularies}
\end{table}

In addition, there are pre-existing vocabularies for many specialised applications.
It's always good to use a pre-existing vocabulary if possible. 

\subsection{Developing an XML vocabulary}

The process is:

\begin{enumerate}
\item List out the possible fields (tag names).
\item Draw a tree diagram.
\item Markup some sample XML data.
\item Factor in potential business rules and add them to our tree diagram. 
\item Optionally formalise the vocabulary in a machine-readable form.
\end{enumerate}

\subsection{Worked example}

As an example, we will develop a vocabulary that would let you specify address data for a person.

\subsubsection{Possible fields}
Start with:
\begin{itemize}
\item firstname
\item secondname
\item address1
\item address2
\item city
\item country
\end{itemize}

Each of these can be specified as tags like
\begin{minted}{xml}
<firstname></firstname>
\end{minted}

But: a person's name is not really part of an address.
What happens if a person has multiple addresses?
So we really need a nested structure:
\begin{minted}{xml}
<person>
  <address>
  </address>
</person>
\end{minted}


\subsubsection{Diagram}

The nested structure of our document is then laid out as a \textbf{tree diagram}, \autoref{fig:tree-diagram}. 

\begin{figure}[htbp]
  \centering
  \begin{tikzpicture}
  \node (person) [cpt] {person};
  \node (firstname) [cpt, below left=of person] {firstname};
  \draw (person) to (firstname);
  \node (secondname) [cpt, below=of person] {secondname};
  \draw (person) to (secondname);
  \node (address) [cpt, below right=of person] {address};
  \draw (person) to (address); 
  \node (street) [cpt, below left=of address] {street};
  \draw (address) to (street); 
  \node (city) [cpt, below=of address] {city};
  \draw (address) to (city); 
  \node (country) [cpt, below right=of address] {country}; 
  \draw (address) to (country); 
  \end{tikzpicture}
  \caption{Tree diagram for address example}
  \label{fig:tree-diagram}
\end{figure}

\subsubsection{Sample data}

\inputminted{xml}{sampledata.xml}

\subsubsection{Rules}

We then decide on a number of business rules to determine if an element is optional, and if there are any constraints on its \textbf{multiplicity}: 

\begin{itemize}
\item Person can have multiple first names.
\item Person can have only one second names.
\item Person can have multiple addresses.
\item An address can have a few streets (lets say between 1 and 3).
\end{itemize}

These rules are then added to the tree diagram, \autoref{fig:tree-diagram-with-rules}. 

\begin{figure}[htbp]
  \centering
  \begin{tikzpicture}
  \node (person) [cpt] {person};
  \node (firstname) [cpt, below left=of person, fill=DarkBlue] {firstname};
  \node [bizr] at (firstname.south) {min=1 max=*};  
  \draw (person) to (firstname);
  \node (secondname) [cpt, below=of person] {secondname};
  \draw (person) to (secondname);
  \node (address) [cpt, below right=of person, fill=DarkBlue] {address};
  \node [bizr] at (address.south) {min=1 max=*};  
  \draw (person) to (address); 
  \node (street) [cpt, below left=of address, fill=DarkGreen] {street};
  \node [bizr] at (street.south) {min=1 max=3};  
  \draw (address) to (street); 
  \node (city) [cpt, below=of address] {city};
  \draw (address) to (city); 
  \node (country) [cpt, below right=of address] {country}; 
  \draw (address) to (country); 
  \end{tikzpicture}
  \caption{Tree diagram with business rules for address example}
  \label{fig:tree-diagram-with-rules}
\end{figure}
 
\section{XPath}

XPath expressions allow us to easily retrieve the contents of elements and attributes in an XML document, by writing expressions.  

The XPath expression language is standardised and works similarly in many languages - Java, C\#, C++/libxml/JavaScript etc.

XPath uses \textbf{expressions} and standard \textbf{functions} to return \textbf{nodes} that are \textbf{related} to other nodes. 


\subsection{Expressions}

XPath uses path expressions to select nodes or node-sets in an XML document.
These expressions are very similar in theory to file paths or URLs:
Examples:
\begin{itemize}
\item \texttt{/Users/peadar/Intro.pdf}
\item \texttt{/Users/peadar/Documents/Intro.pdf}
\item \url{http://xyz.com/dir/file2.xml}
\item \url{http://xyz.com/dir/sd/file2.xml}
\end{itemize}


\subsection{Standard functions}

XPath includes over 100 built-in functions.
There are functions for string values, numeric values, date and time comparison, Boolean values, and more complex manipulations.
  
Functions usually \textbf{operate} on nodes or node-sets returned using an expression. 


\subsection{Nodes}

In XPath, there are seven kinds of nodes:

\begin{enumerate}
\item \textbf{Element}
\item \textbf{Attribute}
\item \textbf{Text}
\item Namespace (see later)
\item Processing instruction 
\item Comment
\item \textbf{Document}
\end{enumerate}


\subsection{Relationships}

Effective use of XPath requires a firm understanding of the relationships present between various nodes, \autoref{tab:relationships}.

\begin{table}[htbp]
  \centering
  \begin{tabularx}{1.0\linewidth}{l X}
    \toprule
    \textbf{Relationship} & \textbf{Meaning} \\
    \midrule
    Parent & Each element and attribute has one parent. \\
    Children & Element nodes may have zero, one or more children. \\
    Siblings & Nodes that have the same parent. \\
    Ancestors & A node's parent, parent's parent, etc. \\
    Descendants & A node's children, children's children, etc. \\
    \bottomrule
  \end{tabularx}
  \caption{Important relationships between nodes}
  \label{tab:relationships} 
\end{table}


\section{Basic selection expressions}

XPath uses path expressions to select nodes or node-sets in an XML document.
The node is selected by following a path or steps.
The most useful path expressions are listed in \autoref{tab:xpath-expressions}.

\begin{table}[htbp]
  \centering
  \begin{tabular}{l l}
    \toprule
    \textbf{Expression} & \textbf{Selects} \\
    \midrule
    \textit{elementname} &  all ``elementname'' elements\\
    / & from root node / previous node \\
    // & from anywhere in the document \\
    .. & parent of the current node \\
    @attribname & attribute ``attribname'' \\
    text() & text content of current node\\
    \bottomrule
  \end{tabular}
  \caption{XPath expressions}
  \label{tab:xpath-expressions}
\end{table}

\subsection{Writing path expressions}

A location path can be absolute or relative.
An absolute location path starts with a slash ( / ) and a relative location path does not.
In both cases the location path consists of one or more steps, each separated by a slash:

\begin{verbatim}
An absolute location path:

/step/step/...

A relative location path:

step/step/...
\end{verbatim}

Each step is evaluated against the nodes in the current node-set.


\subsection{Example}

Using the bookstore document, a number of XPath expressions are shown in \autoref{tab:expressions-example}. 

\begin{table}[htbp]
  \centering
  \begin{tabular}{l l}
    \toprule
    \textbf{Expression} & \textbf{Selects} \\
    \midrule
    \texttt{bookstore} & all elements named ``bookstore'' \\
    \texttt{/bookstore} & root element bookstore \\
    \texttt{bookstore/book} & child books of bookstore \\
    \texttt{//book} & all books in document \\
    \texttt{bookstore//book} & descendant books of bookstore\\
    \bottomrule
  \end{tabular}
  \caption{Expressions on \texttt{bookstore.xml} document}
  \label{tab:expressions-example}
\end{table}


\subsubsection{Selecting text content}

Note that we must use the \texttt{text()} function to get just the text content of an element, rather than the element itself:
\begin{itemize}
\item \texttt{/bookstore/book/title} = title element of each book
\item \texttt{/bookstore/book/title/text()} = title text content of each book
\end{itemize}

\section{Tooling}

\subsection{Command-line}

The \texttt{xmllint} tool can evaluate XPath expressions at the command-line.

\begin{minted}{bash}
xmllint --xpath '/route/station[amenity]' stations.xml
\end{minted}


\subsection{Python}

Python's \texttt{lxml} library can work with XPath expressions:
\begin{minted}{python}
import lxml.etree as ET

root = ET.parse('stations.xml')
for name in root.xpath('/route/station/name/text()'):
    print(name)
\end{minted}



\section{Predicates}

Predicates are used to find a \textbf{specific node} or a \textbf{node containing a specific value}.

They are embedded in square brackets after the element name to which they refer.
Examples:
\begin{itemize}
\item \texttt{/element/element[predicate]}
\item \texttt{/element[predicate]/element}
\item \texttt{/element[p1]/element[p2]}
\end{itemize}


\subsection{Posessive predicates}

Posessive predicates select elements that have matching children, \autoref{tab:posessive-predicates}. 

\begin{table}[htbp]
  \centering
  \begin{tabularx}{1.0\linewidth}{l X}
    \toprule
    \textbf{Expression} & \textbf{Selects} \\
    \midrule
    \textbf{//parent[@attribute]} &  all ``parent'' elements with  ``attribute'' \\
    \textbf{//gp/parent[child]} & all ``parent'' elements with ``child'' \\
    \textbf{//gp/parent[child]/child2} &  all ``child2'' of above... \\
    \bottomrule
  \end{tabularx}
  \caption{Posessive predicates}
  \label{tab:posessive-predicates}
\end{table}

\subsection{Value-based predicates}

Value-based predicates select elements based on whether they or their child elements or attributes match specific values, \autoref{tab:value-based-predicates}. 

  \begin{table}[htbp]
    \centering
    \begin{tabularx}{1.0\linewidth}{l X}
      \toprule
      \textbf{Expression} & \textbf{Selects} \\
      \midrule
      \texttt{//parent[@attribute='value']} &  all ``parent'' elements with ``attribute''==value \\
      \texttt{//gp/parent[child$<v$]} &  all ``parent'' elements where ``child'' element has value $<v$ \\
      \texttt{//gp/parent[child$<v$]/child2} &  all ``child2'' of above... \\
      \bottomrule
    \end{tabularx}
    \caption{Value-based predicates}
    \label{tab:value-based-predicates}
  \end{table}

\subsection{Positional predicates}

Positional predicates allow selections based on element ordering usually using XPath functions, \autoref{tab:positional-predicates}. 

\begin{table}[htbp]
  \centering
  \begin{tabularx}{1.0\linewidth}{l X}
    \toprule
    \textbf{Expression} & \textbf{Selects} \\
    \midrule
    /parent/child[$n$] & the $n$th child element of the parent. \\
    /parent/child[last()] & the last child element of the parent. \\
    /parent/child[last()-$n$] & the last but $n$ child element of the parent. \\
    /parent/child[position()$<n$] & the first $n-1$ child elements of the parent. \\
    \bottomrule
  \end{tabularx}
  \caption{Positional predicates}
  \label{tab:positional-predicates}
\end{table}

\subsection{Other predicates}

There are a lot more things you can do with predicates --- see online tutorials for more details \cite{w3-schools:2013:xpath}. 

\subsection{Combining two node sets}

To combine two node sets, we use the $\vert$ (vertical bar) operator between two separate XPath expressions.
Example: 

\begin{verbatim}
    \texttt{/path/path[predicate] $\vert$ /path2/path2[predicate]}
\end{verbatim}



\subsection{Operators}

When writing a predicate we can use any of the usual binary comparison operators, \autoref{tab:operators}.

\begin{table}[htbp]
  \centering
  \begin{tabular}{l l}
    \toprule
    \textbf{Operator} & \textbf{Meaning} \\
    \midrule
    = & equal \\
    != & not equal \\
    $<, >$ & less than, greater than \\
    $<=, >=$ & less/greater than or equal to \\
    or, and & or/and \\
    \bottomrule
  \end{tabular}
  \caption{XPath operators}
  \label{tab:operators}
\end{table}

You can use math operators in your comparisons, but \textbf{be careful of division}, \autoref{tab:math-operators}. 

\begin{table}[htbp]
  \centering
  \begin{tabular}{l l}
    \toprule
    \textbf{Operator} & \textbf{Meaning} \\
    \midrule
    $+$ & addition \\
    $-$ & subtraction \\
    * & multiplication \\
    div & division \\
    mod & modulus / division remainder\\
    \bottomrule
  \end{tabular}
  \caption{XPath operators}
  \label{tab:math-operators}
\end{table}


\section{Wildcards}

XPath wildcards can be used to select unknown XML elements, \autoref{tab:xpath-wildcards}

\begin{table}[htbp]
  \centering
  \begin{tabular}{l l}
    \toprule
    \textbf{Expression} & \textbf{Matches} \\
    \midrule
    \texttt{*} &  any element node \\
    \texttt{@*} &  any attribute node \\
    \texttt{node()} & any node of any kind \\
    \bottomrule
  \end{tabular}
  \caption{XPath wildcards}
  \label{tab:xpath-wildcards}
\end{table}





\section{Axes}

An axis defines a node-set relative to the current node.

\subsection{Direct axes}

Direct axes specify the node itself, its parent or children, or its attributes, \autoref{tab:direct-axes}. 

\begin{table}[htbp]
  \begin{tabularx}{1.0\linewidth}{l X}
    \toprule
    \textbf{Axis} & \textbf{Selects} \\
    \midrule
    self & the current node \\
    parent & the parent of the current node \\
    child & all children of the current node \\
    attribute & all attributes of the current node \\
    \bottomrule
  \end{tabularx}
  \caption{Direct axes}
  \label{tab:direct-axes}
\end{table}


\subsection{Cross-generational axes} 

Cross-generational axes allow us to specify nodes ancestral or descending nodes, \autoref{tab:cross-generational-axes}. 

\begin{table}[htbp]
  \begin{tabularx}{1.0\linewidth}{l X}
    \toprule
    \textbf{Axis} & \textbf{Selects} \\
    \midrule
    ancestor & all ancestors (parent, grandparent, etc.) of the current node \\
    ancestor-or-self & current node and all of its ancestors \\
    descendant & all descendants (children, grandchildren, etc.) of the current node \\
    descendant-or-self & current node and all descendents \\
    \bottomrule
  \end{tabularx}
  \caption{Cross-generational axes}
  \label{tab:cross-generational-axes}
\end{table}




\subsection{Positional axes} 

Positional axes allow us to select nodes before or after other nodes in the document, \autoref{tab:positional-axes}. 

\begin{table}[htbp]
  \begin{tabularx}{1.0\linewidth}{l X}
    \toprule
    \textbf{Axis} & \textbf{Selects} \\
    \midrule
    following & everything in the document after the closing tag of the current node \\
    following-sibling & all siblings after the current node \\
    preceding &  all nodes  before  current node, except ancestors, attribute nodes and namespace nodes \\
    preceeding-sibling & all siblings before the current node \\
    \bottomrule
  \end{tabularx}
  \caption{Positional axes}
  \label{tab:positional-axes}
\end{table}

\subsection{Writing location path expressions}

With predicates and axes, a full step consists of:

\begin{itemize}
\item an axis (defines the tree-relationship between the selected nodes and the current node)
\item a node-test (identifies a node within an axis)
\item zero or more predicates (to further refine the selected node-set)
\end{itemize}

The syntax for a location step is:
\begin{verbatim}
axisname::nodetest[predicate]
\end{verbatim}


\subsection{Example}

Here are some XPath expressions using axes for the bookstore.xml file:

\begin{table}[htbp]
  \begin{tabularx}{1.0\linewidth}{l X}
    \toprule
    \textbf{Expression} & \textbf{Selects} \\
    \midrule
    child::book & 	 all book nodes that are children of the current node \\
    attribute::lang &	 the lang attribute of the current node \\
    child::* &	 all element children of the current node \\
    attribute::* &	 all attributes of the current node \\
    child::text() &	 all text node children of the current node \\ 
    child::node() &	 all children of the current node \\ 
    descendant::book &	 all book descendants of the current node \\ 
    ancestor::book &	 all book ancestors of the current node \\ 
    ancestor-or-self::book &	 all book ancestors of the current node - and the current as well if it is a book node \\
    child::*/child::price &	 all price grandchildren of the current node\\
    \bottomrule
  \end{tabularx}
  \caption{XPath axes expressions example}
  \label{tab:xpath-axes-example}
\end{table}


\section{XQuery}

XQuery is to XML what SQL is to database tables.
XQuery was designed to \textbf{query} rather than \textit{navigate} XML data.

\subsection{Syntax}

XQuery is built on top of \textbf{XPath}, and uses the \textbf{same expressions and predicates},  
augmented by the queries made up of by the so-called \textbf{FLWOR} syntax, \autoref{tab:flwor}.

\begin{table}[htbp]
  \begin{tabularx}{1.0\linewidth}{l X}
    \toprule
    \textbf{Component} & \textbf{Function} \\
    \midrule
    \textbf{f}or & creates a sequence of nodes \\
    \textbf{l}et & binds a sequence to a variable \\
    \textbf{w}here & filters nodes on a boolean expression \\
    \textbf{o}rder by & sorts the nodes \\
    \textbf{r}eturn & evaluated once per node \\
    \bottomrule
  \end{tabularx}
  \caption{FLWOR syntax}
  \label{tab:flwor}
\end{table}


\subsection{Rules}

\begin{enumerate}
\item XQuery is case-sensitive
\item XQuery elements, attributes, and variables must be valid XML names
\item An XQuery string value can be in single or double quotes
\item Local variables (usually to represent the current node) are preceeded by the \$ character. 
\item XQuery comments are delimited by (: and :), e.g. (: XQuery Comment :)
\end{enumerate}


\subsection{Examples}

We will use the following XML document:

\inputminted{xml}{books.xml}


\subsubsection{Basic selection}

We are asked to:
\begin{quotation}
Select all the title elements under the book elements that are under the bookstore element that have a price element with a value that is higher than 30.
\end{quotation}

Write down a suitable XPath expression for this: 

\vspace{\fill}

This same query could be expressed using the FLWOR syntax as follows:

\begin{minted}{xquery}
for $x in /bookstore/book
where $x/price > 30.00
return $x/title
\end{minted}

Obviously, the XPath expression is more concise. 
However, we have a problem if we want the returned element list to be ordered or otherwise modified:

\begin{minted}{xquery}
for $x in /bookstore/book
where $x/price > 30.00
order by $x/title
return $x/title
\end{minted}

We can simplify by using XPath predicates instead of (or in addition to) the WHERE clause: 

\begin{minted}{xquery}
for $x in /bookstore/book[price>30]
order by $x/title
return $x/title
\end{minted}

\subsection{Counting loop iterations}

The at keyword can be used to count the iteration:

\begin{minted}{xquery}
for $x at $i in /bookstore/book/title
return {$i}. {data($x)}
\end{minted}

\subsection{Multiple order by clauses}

The order by clause is used to specify the sort order of the result. Here we want to order the result by category and title:

\begin{minted}{xquery}
for $x in /bookstore/book
order by $x/@category, $x/title
return $x/title
\end{minted}

\bibliographystyle{plainnat}
\bibliography{bibliography}

\end{document}



