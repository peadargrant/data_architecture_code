\chapter{UNIX tools}

\section{UNIX tools}\label{unix-tools}

There are a few useful non-SQL tools we will need. Many of these tools work (or have equivalents) on Windows desktops too.
The best way to learn how to use these is to use their inbuilt help or man pages.

Inbuilt help is usually accessed by a switch, often \texttt{-h}.
If a command is run with no input it will often give the help command in the error message.

\subsection{Manpages}

Most tools also have a manual page called a manpage:

\begin{verbatim}
# general pattern
man commandname

# example man page for curl 
man curl
\end{verbatim}

Hit \texttt{q} to exit a manpage.

\subsection{Printing file contents to screen}
\label{printing-file-contents-to-screen}

The \texttt{cat} command will print the contents of a file to the screen.
We can use \texttt{more} to paginate it.

\begin{verbatim}
# listing the contents of the current directory
ls

# printing out the stations.csv file to the screen
cat stations.csv

# printing out the file with pagination
cat stations.csv | more
# using the bar | operator pipes the output of one command to another
\end{verbatim}

\section{Text editors}

UNIX systems have a number of text editors available.
Most popular ones are \texttt{nano}, \texttt{emacs} and \texttt{vim}.

\subsection{nano}

Nano is a simple lightweight editor.

\begin{minted}{bash}
# start nano in a new (unnamed) file
nano

# open an existing or named new file
nano myfile.txt
\end{minted}

When open it shows shortcuts at the bottom of the screen.

\subsection{emacs}

Emacs is a more fully-featured editor that has a huge range of capabilities.
It's more akin to a fully featured IDE like Eclipse, except in text mode.
It has support for many languages.
Basic usage is same as nano:

\begin{minted}{bash}
# start emacs in a new (unnamed) file
emacs

# open an existing or named new file
emacs myfile.txt
\end{minted}

Emacs has pulldown menus accessible with the F10 key by default.
Other useful commands.

\begin{verbatim}
# save file
Ctrl-X s

# quite emacs
Ctrl-X Ctrl-C
\end{verbatim}

Emacs can handle multiple files simultaneously:

\begin{verbatim}
# open a new file
Ctrl-X Ctrl-f

# Close the current file
Ctrl-X k

# Switch to another tab
Ctrl-X (right arrow key)
Ctrl-X (left arrow key)
\end{verbatim}

\section{Chat server}

We have become used to the ``chat window'' in Zoom, Teams etc.
To help share commands etc during class I have setup a chat facility on the shared server.
Type \texttt{chat} to start it.

Once you are connected to the channel you can type messages into the chatroom.
Type \texttt{/QUIT} to quit.

Internally this uses a local Internet Relay Chat (IRC) server and the \texttt{irssi} full-screen IRC client.
There is a single channel \texttt{\#dataarch}. 
That \texttt{chat} command sets up a simple configuration file to launch \texttt{irssi} directly into the \texttt{\#dataarch} topic. 

\section{Multitasking}

In a command-line environment over SSH we are restricted to running one application at a time.
This can be problematic when we want to concurrently run a number of programs, e.g. psql, a text editor and the chat window. 

You could get around this by logging in multiple times concurrently from your client.
Instead we can use a terminal multiplexer called \texttt{tmux} to let us concurrently run multiple sessions and switch between them.

\subsection{Starting tmux}

I have configured the shared server so that it by default starts \texttt{tmux} on login.
This isn't the case in most cases with Linux / UNIX but it can be very useful.
(Tmux can be started by typing \texttt{tmux} which will spawn a new tmux session with a single window.)

\subsection{Window handling}

Once in a session, \texttt{tmux} is quite powerful and featured.
It has many commands that are accessed by typing a prefix, by default \texttt{Ctrl-B} followed by another key.

\begin{verbatim}
# open a new window
Ctrl-B c

# name the current window
Ctrl-B , 

# switch to a numbered window 
Ctrl-B 2

# kill the current window
Ctrl-B x 
\end{verbatim}

\subsection{Detach}

Tmux sessions can be detached and then attached again.
When in a session we can detach:
\begin{verbatim}
# detach 
Ctrl-B d 
\end{verbatim}

We can then attach the session again:
\begin{verbatim}
# attach
tmux attach
\end{verbatim}

The session is held open for us, even if we log out or lose connectivity.
We can actually attach to the same session from multiple devices.
The session is also detached if we lose connectivity or just terminate the SSH session.

\subsection{Using CURL to download data}
\label{using-curl-to-download-data}

When we are logged into a remote server we may wish to download data
from the internet. There are actually text-mode browsers like lynx,
links, links2 but these have become less usable since the web's descent
into excessive visual flourish and JavaScript-centricity.

Instead we will make use of a tool called \texttt{curl} which can
download the contents of a particular URL to a file. CURL is a very
useful tool, and you should consult its inbuilt help or manpages
(\texttt{man\ curl}) to learn more about its features. Basic usage:

\begin{verbatim}
# general pattern
curl -o output_file_name url

# example, download list of train stations in London
# https://www.doogal.co.uk/london_stations.php
# https://www.doogal.co.uk/LondonStationsCSV.ashx
curl -o stations.csv  https://www.doogal.co.uk/LondonStationsCSV.ashx
\end{verbatim}


