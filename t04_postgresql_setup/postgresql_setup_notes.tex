\chapter{PostgreSQL setup}
\label{ch:postgresql-setup}

These instructions install the PostgreSQL DBMS locally on our own computer.
This is very handy for development and personal projects.
This WILL NOT work on Lab Desktop machines!

The instructions here assume you are using a Windows computer.
Follow the instructions further down the page for Mac or Linux.

\section{Windows}
\label{sec:postgresql-setup-windows}

\subsection{PostgreSQL Server Installation}

\begin{enumerate}

\item
	Download the PostgreSQL 16 installer from:
	\url{https://www.enterprisedb.com/downloads/postgres-postgresql-downloads}

\item
	Double click to start the installation.
	You should see \textit{Welcome to the PostgreSQL Setup Wizard}.
	Click Next.
	
\item 
	Leave the \textbf{Installation} Directory at the default.
	Click next.
	
\item In \textbf{Select Components} leave all selected.
	Click next.
	
\item In \textbf{Data Directory} leave all selected.
	Click next.
	
\item In \textbf{Password} you need to provide a password for the database superuser.
	PostgreSQL (and other DBMSs) have internal usernames / passwords for access control (we will meet later on).
	It is best NOT to use your computers' or DkIT password.
	Note the password down in a text file on your computer.
	Click next.

\item In \textbf{Port} leave it at 5432.
	Click next.
	
\item In \textbf{Advanced Options} leave \textit{Locale} at its default.
	Click next.

\item \textbf{Pre-installation} summary should be similar to:
\begin{verbatim}
Installation Directory: C:\Program Files\PostgreSQL\16
Server Installation Directory: C:\Program Files\PostgreSQL\16
Data Directory: C:\Program Files\PostgreSQL\16\data
Database Port: 5432
Database Superuser: postgres
Operating System Account: NT AUTHORITY\NetworkService
Database Service: postgresql-x64-16
Command Line Tools Installation Directory: C:\Program Files\PostgreSQL\16
pgAdmin4 Installation Directory: C:\Program Files\PostgreSQL\16\pgAdmin 4
Stack Builder Installation Directory: C:\Program Files\PostgreSQL\16
Installation Log: C:\Users\grantp\AppData\Local\Temp\install-postgresql.log
\end{verbatim}

\item On the \textbf{Ready to Install} screen click Next.
	The installation process will take a few minutes.
	
\item Finally you will see \textit{Completing the PostgreSQL setup wizard}.
	You will see \textit{Launch Stack Builder at Exit}.
	Turn this off.
	Click Finish.
	
%\item \textbf{Stack Builder} will open then. %
%	Drop the menu box down to \texttt{PostgreSQL 16 (x64) on port 5432}.
%	Click next.

\end{enumerate}
	
\subsection{PATH setup}

\begin{enumerate}

\item
	Search for \textit{Advanced System Settings} in the start menu.
	
\item 
	Click the \textit{Advanced} tab at the top.
	
\item 
	Click the \textit{Environment Variables} button.
	
\item 
	Under \textit{User Variables} click \textit{PATH} and then textit{Edit}.
	
\item 
	Click \textit{New} and type exactly:\\
\begin{verbatim}
	C:\Program Files\PostgreSQL\16\bin
\end{verbatim}

\item 
	Click \textit{OK} to leave this box, \textit{OK} again to leave the Environment Variables box and \textit{OK} a third time to leave the Advanced System Settings box.

\item
	Open a New PowerShell windows.
	Type \texttt{psql} and press enter.
	It should request a password, just Ctrl-C to exit it.

\end{enumerate}

\subsection{PostgreSQL User setup}

\begin{enumerate}

\item
	Open a New PowerShell window.

\item
	Type \texttt{createuser -s -P -U postgres your-windows-username-here}.
	
\item
	Enter the superuser password you stored earlier.

\item
	Type Ctrl-C to quit.
	Then type just:
	\texttt{psql}
	and enter your PostgreSQL password.
	You should see an error like:
\begin{verbatim}
psql: error: connection to server at "localhost" (::1),
port 5432 failed: FATAL:  database "grantp" does not exist
\end{verbatim}
	PostgreSQL expects a database by default to exist with the same name as the user.

\item
	Type \texttt{createdb your-windows-username-here}, e.g.
	\texttt{createdb grantp}.
	
\item 
	Now try \texttt{psql} again, enter your password and you should be connected to your database.
	To confirm type \texttt{SELECT 1;} and press enter.
	You should see something like:
\begin{verbatim}
 ?column?
----------
        1
(1 row)
\end{verbatim}

\end{enumerate}


\section{Mac}
\label{sec:postgresql-setup-mac}

\section{Linux}
\label{sec:postgresql-setup-linux}

