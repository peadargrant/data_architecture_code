\chapter{PostgreSQL intro}

\section{Database basics}\label{database-basics}

Make sure you can connect to the shared SSH server.

\subsection{Running psql}\label{running-psql}

Once logged in, we can use the \texttt{psql} command-line client to connect to the PostgreSQL server.
Each server can host a number of databases independently of each other.
Today we are going to use a database called \texttt{population}.

\begin{verbatim}
psql population
\end{verbatim}

Note the DBMS didn't ask us for a username/password.
This is because it is setup to trust local connections with the same name as the login username.
Generally a DBMS will ask for a username/password.

Once logged in you should see a prompt similar to:

\begin{verbatim}
psql (9.2.24)
Type "help" for help.

population=> 
\end{verbatim}

There are two types of command we will encounter:

\begin{itemize}
\item
  SQL statements to be executed by the server
\item
  Client commands (that sometimes get expanded into server-side SQL).
  These are usually prefixed with a backslash.
\end{itemize}

You can get a list of all client commands using \textbackslash{}?.

\subsection{Listing all tables}\label{listing-all-tables}

A database in its simplest form is like an Excel workbook.
Just as an excel workbook has one or more sheets, a database may have one or more tables in it.
To find out the tables (and other objects) in our DB we can ask the database to describe the tables:

\begin{verbatim}
\dt
\end{verbatim}

Our population database has 1 table:

\begin{verbatim}
           List of relations
 Schema |    Name    | Type  |  Owner   
--------+------------+-------+----------
 public | population | table | ec2-user
(2 rows)
\end{verbatim}

\subsection{Describing a table}\label{describing-a-table}

A table consists of a number of columns. Each row is a record. We can
find out the schema definition using the client command
\texttt{tablename}. To find out more about the \texttt{population} table
we could say:

\begin{verbatim}
\d population
\end{verbatim}

\href{https://www.postgresql.org/docs/13/datatype.html}{Refer to
datatypes in the PostgreSQL manual}

\section{SELECT queries}\label{select-queries}

We will review basic querying operations in SQL.
If you have experience already with relational databases (MySQL, Microsoft SQL, Oracle) you will find most of this section familiar.
You should refer to the \href{https://www.postgresql.org/docs/13/queries.html}{queries section of the PostgreSQL manual} for further detail on the examples below.

\subsection{Entire table}\label{entire-table}

Selecting an entire table using the asterisk (*) to pick all columns.

\begin{verbatim}
select * from population ;
\end{verbatim}

Note that statements need to be terminated with the semicolon (;). If
you omit the semicolon the prompt will change from
\texttt{=\textgreater{}} to \texttt{-\textgreater{}}.

In PostgreSQL we can also use the shortened version

\begin{verbatim}
table directions;
\end{verbatim}

\subsection{Sorting}\label{sorting}

Unlike a spreadsheet, the order of rows in result sets from database
queries is non-deterministic by default. They are \emph{not} necessarily
alphabetical order, insertion order or any other pattern. We use the
\texttt{ORDER\ BY} clause to enforce ordering:

The order direction is either \texttt{asc} for ascending or
\texttt{desc} for descending.

See \href{https://www.postgresql.org/docs/13/queries-order.html}{sorting
page of PostgreSQL manual} for further details.

\subsection{Where clause}\label{where-clause}

The \texttt{WHERE} clause limits the output


Text is best compared using the \texttt{LIKE} operator which allows
\texttt{\%} to represent any text.


\subsection{Column specification}\label{column-specification}

We can specify specific column names


\subsection{Column expressions}\label{column-expressions}

PostgreSQL supports powerful expressions that combine the values of
different columns. The columns themselves need not be selected in the
result set.


\subsection{Column labels}\label{column-labels}

Columns in the result set receive automatic names. These are normally
the name of the column or are derived from an expression. We can assign
specific names using the \texttt{AS} keyword.


See
\href{https://www.postgresql.org/docs/13/queries-select-lists.html\#QUERIES-COLUMN-LABELS}{column
labels in PostgreSQL manual}.

\section{Aggregate functions}\label{aggregate-functions}

Mastering aggregate functions will significantly improve your
capabilities to query databases. You should refer to the
\href{https://www.postgresql.org/docs/13/functions-aggregate.html}{aggregate
functions section of the PostgreSQL manual} as you review the following
examples.

\subsection{Count}\label{count}

The simplest aggregate function is \texttt{COUNT} to return the number
of rows matched by a specific query. At its simplest, we can tell the
number of rows in a table.

\begin{verbatim}
select count(*) from population; 
\end{verbatim}

Count is often used with where to obtain the number of rows matching a
specific condition.




